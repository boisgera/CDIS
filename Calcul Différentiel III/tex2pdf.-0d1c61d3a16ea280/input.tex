\PassOptionsToPackage{unicode=true}{hyperref} % options for packages loaded elsewhere
\PassOptionsToPackage{hyphens}{url}
%
\documentclass[french,]{article}
\usepackage{lmodern}
\usepackage{amssymb,amsmath}
\usepackage{ifxetex,ifluatex}
\usepackage{fixltx2e} % provides \textsubscript
\ifnum 0\ifxetex 1\fi\ifluatex 1\fi=0 % if pdftex
  \usepackage[T1]{fontenc}
  \usepackage[utf8]{inputenc}
  \usepackage{textcomp} % provides euro and other symbols
\else % if luatex or xelatex
  \usepackage{unicode-math}
  \defaultfontfeatures{Ligatures=TeX,Scale=MatchLowercase}
\fi
% use upquote if available, for straight quotes in verbatim environments
\IfFileExists{upquote.sty}{\usepackage{upquote}}{}
% use microtype if available
\IfFileExists{microtype.sty}{%
\usepackage[]{microtype}
\UseMicrotypeSet[protrusion]{basicmath} % disable protrusion for tt fonts
}{}
\IfFileExists{parskip.sty}{%
\usepackage{parskip}
}{% else
\setlength{\parindent}{0pt}
\setlength{\parskip}{6pt plus 2pt minus 1pt}
}
\usepackage{hyperref}
\hypersetup{
            pdftitle={Calcul Différentiel II},
            pdfauthor={STEP, Mines Paristech},
            pdfborder={0 0 0},
            breaklinks=true}
\urlstyle{same}  % don't use monospace font for urls
\setlength{\emergencystretch}{3em}  % prevent overfull lines
\providecommand{\tightlist}{%
  \setlength{\itemsep}{0pt}\setlength{\parskip}{0pt}}
\setcounter{secnumdepth}{0}
% Redefines (sub)paragraphs to behave more like sections
\ifx\paragraph\undefined\else
\let\oldparagraph\paragraph
\renewcommand{\paragraph}[1]{\oldparagraph{#1}\mbox{}}
\fi
\ifx\subparagraph\undefined\else
\let\oldsubparagraph\subparagraph
\renewcommand{\subparagraph}[1]{\oldsubparagraph{#1}\mbox{}}
\fi

% set default figure placement to htbp
\makeatletter
\def\fps@figure{htbp}
\makeatother

\usepackage{fontawesome}

\usepackage{bookmark}
\ifnum 0\ifxetex 1\fi\ifluatex 1\fi=0 % if pdftex
  \usepackage[shorthands=off,main=french]{babel}
\else
  % load polyglossia as late as possible as it *could* call bidi if RTL lang (e.g. Hebrew or Arabic)
  \usepackage{polyglossia}
  \setmainlanguage[]{french}
\fi

\title{Calcul Différentiel II}
\author{STEP, Mines Paristech\footnote{Ce document est un des produits du projet
  \href{https://github.com/}{\(\mbox{\faGithub}\)
  \texttt{boisgera/CDIS}}, initié par la collaboration de
  \href{mailto:sebastien.boisgerault@mines-paristech.fr}{(S)ébastien
  Boisgérault} (CAOR),
  \href{mailto:thomas.romary@mines-paristech.fr}{(T)homas Romary} et
  \href{mailto:emilie.chautru@mines-paristech.fr}{(E)milie Chautru}
  (GEOSCIENCES),
  \href{mailto:pauline.bernard@mines-paristech.fr}{(P)auline Bernard}
  (CAS), avec la contribution de
  \href{mailto:gabriel-stolz@mines-paristech.fr}{Gabriel Stoltz} (Ecole
  des Ponts ParisTech, CERMICS). Il est mis à disposition selon les
  termes de \href{http://creativecommons.org/licenses/by-nc-sa/}{la
  licence Creative Commons ``attribution -- pas d'utilisation
  commerciale -- partage dans les mêmes conditions'' 4.0
  internationale}.}}
\date{12 août 2019}

\begin{document}
\maketitle

{
\setcounter{tocdepth}{3}
\tableofcontents
}
TODO, à trier, idées en vrac:

\begin{itemize}
\item
  e.v.n fonctionnels (\(C^k\), \(L^1\), \(L^2\), etc.), Hilbert, Banach.
\item
  opérateurs linéaires continus (bornés), adjoint (appli calcul diff,
  par exemple pour calcul solution diff EDP sans recalcul ?)
\item
  différentielle (Fréchet), cas général
\end{itemize}

En fait va plus loin que le scope du calcul diff, contient un mini
``topo en dim infinie'' (qu'est-ce qui change ? Compacité (en
particulier ppté vraie sur les compacts ne donne plus le semi-global,
ppté de type Ascoli-Arzela pour caractériser la compacité pour les fcts
continues), non-équivalence des normes, dualité, topologie faible ?) et
algèbre linéaire en dim infinie (opérateurs linéaires ne sont pas tous
cont., pas de chance \ldots{} Hilbert, Banach, etc).

Prérequis: Topo, Calcul Diff dim finie, sans doute l'intégrale (les
critères intégraux sont une grande motivation pour enseigner le calcul
diff en dim infinie), c'est aussi sans doute l'endroit ou on veut parler
de la complétude des espaces comme \(L^1 / L^2\), etc.

Exemples à traiter: ``interpolation'' données non-paramétrique, calcul
des variations, maximum entropie sous contrainte, diff. / chemin (ex:
usage en analyse complexe), optimisation/gradient forme, etc. Autre
exemple: quantif qui optimise le SNR ou l'entropie. Autre exemple: Pb de
Dirichlet variationel ? On peut faire ça ? Et lier ça au Laplacien ? Ou
il il faut un cadre compliqué (trace \& co) ?

Articulation avec Physique Fonda. et Appliquée (calc var, Hilb, etc.).
Volonté de permettre de comprendre des trucs comme la construction de
Fourier (prolgt opé lin con à partir d'un ensemble dense avec
majoration), ou typiquement, définition de la trace sur un bord régulier
\ldots{}

\begin{itemize}
\item
  Topo en dim infinie, Banach, Hilbert, opérateurs lin cont, analyse
  spectrale
\item
  Calcul diff en dim infinie (acc. fini, cont df, inversion locale,
  point critique et multiplicateurs de lagrange, etc., tout ça revisité
  rapidement en se basant sur la familiarité avec la dim finie, déjà
  vue).
\item
  Equation de Poisson: intro Sobolev, pb ``variationnel'' en
  multivariable, trace, etc. Evocation schéma résolution numérique (élt
  finis) ?
\item
  Cadre Méca Q, opérateurs (non bornés) hermitien, semi-groupes
  (unitaires) fortement continus, etc?
\end{itemize}

\newcommand{\N}{\mathbb{N}}
\newcommand{\Z}{\mathbb{Z}}
\newcommand{\Q}{\mathbb{Q}}
\newcommand{\R}{\mathbb{R}}
\renewcommand{\C}{\mathbb{C}}

\begin{center}\rule{0.5\linewidth}{\linethickness}\end{center}

\hypertarget{todo-espaces-de-hilbertbanach}{%
\section{TODO -- Espaces de
Hilbert/Banach}\label{todo-espaces-de-hilbertbanach}}

\(L^1\), \(L^2\), \(L^{\infty}\) (\(L^p\) ?) Dans le cadre général ?
(ens de départ ?)

\(C^0\), \(C^k\), etc. Hölder \(C^{k,\alpha}\) ? Sur sous-ensemble
compact de \(\mathbb{R}^n\) ?

Sobolev (sur \(\mathbb{R}^n\)) ? Sur \(\Omega\) par densité ? (ouch)

Espaces d'opérateurs: \(E \stackrel{\ell}{\to } F\).

Opérateurs bornés, cas particulier des ``inclusions'' (fonction
identité, différente norme au départ et à l'arrivée)

Exemple d'opérateurs bornés (ici et en exercice): les identités
(\(C^0 \to L^1\), \(L^2 \to L^1\)), l'intégrale
(\(L^1 \to \mathbb{R}^n\))

Technique de définition d'un opérateur borné par densité.

Liste de sous-ensembles denses (en particulier, fcts lisses à support
compact \ldots{})

Exemple techniques densité pour définir un truc ? et/ou étendre une
relation (exemple: IPP pour fcts avec dérivées faibles ? Fourier ?)

Opérateurs définis sur un domaine \(D(A)\) dense (certains bornés,
d'autres non). Applis méca Q ?

\hypertarget{diffuxe9rentielle}{%
\section{Différentielle}\label{diffuxe9rentielle}}

\hypertarget{diffuxe9rentielle-de-fruxe9chet}{%
\subsubsection{Différentielle de
Fréchet}\label{diffuxe9rentielle-de-fruxe9chet}}

Soient \(E\) et \(F\) deux espaces vectoriels normés et \(U\) un ouvert
de \(E\). La fonction \(f: U \to F\) est \emph{différentiable en
\(x \in U\)} s'il existe une application linéaire continue, notée
\(df(x)\) et appellée \emph{différentielle de \(f\) en \(x\)}, telle que
\[
f(x+h) = f(x) + df(x) \cdot h + o(\|h\|_E),
\] c'est-à-dire si \[
\lim_{h \to 0} \frac{\|f(x+h) - f(x) - df(x) \cdot h\|_F}{\|h\|_E} = 0.
\]

\hypertarget{todo}{%
\subsubsection{TODO}\label{todo}}

Reprendre les règles de calcul de la dimension finie

\hypertarget{TFI-2}{%
\subsubsection{Théorème des Fonctions Implicites}\label{TFI-2}}

Soient \(E\), \(F\) et \(G\) trois espaces vectoriels normés, \(F\)
étant complet, et \(f\) une fonction définie sur un ouvert \(W\) de
\(E \times F\) \[
f: (x, y) \in W \subset \mathbb{R}^n \times \mathbb{R}^m \to f(x, y) \in \mathbb{R}^m
\] qui soit continûment différentiable et telle que la différentielle
partielle \(\partial_y f\) soit inversible et d'inverse continu en tout
point de \(W\). Si le point \((x_0, y_0)\) de \(W\) vérifie
\(f(x_0, y_0)= 0\), alors il existe des voisinages ouverts \(U\) de
\(x_0\) et \(V\) de \(y_0\) tels que \(U \times V \subset W\) et une
fonction implicite \(\psi: U \to F\), continûment différentiable, telle
que pour tous \(x \in U\) et \(y \in V\), \[
f(x, y) = 0
\; \Leftrightarrow \; 
y = \psi(x).
\] De plus, la différentielle de \(\psi\) est donnée pour tout
\(x \in U\) par \[
d \psi(x) = - (\partial_y f(x, y))^{-1} \cdot \partial_x f(x, y) \, \mbox{ où } \, y=\psi(x).
\]

\vspace{3.25ex plus 1ex minus .2ex}\protect\hypertarget{duxe9monstration-esquisse}{}{\textbf{Démonstration
(esquisse)}\quad}La démonstration est similaire au cas de la dimension
finie: la construction de la solution \(f(x, y) = 0\) en \(y\) est
toujours basée sur la construction d'une suite d'approximations \(y_k\)
par la méthode (modifiée) de Newton. L'existence d'un point fixe à cette
méthode repose sur le point fixe de Banach -- d'où l'hypothèse de
complétude de \(F\) -- et la caractère contractant de la fonction de
Newton, qui exploite la norme d'opérateur
\(\|(\partial_y f(x_0, y_0))^{-1}\|\) et donc la continuité de cet
inverse.\hfill$\blacksquare$

\hypertarget{formulation-alternative}{%
\subsubsection{Formulation alternative}\label{formulation-alternative}}

On remarque que sous les hypothèses du théorème des fonctions
implicites, l'espace vectoriel normé \(G\) est nécessairement complet.
En effet, pour toute suite de Cauchy \(z_k\) dans \(G\), comme
\(Q := \partial_y f(x_0, y_0)\) est inversible et d'inverse continu,
\(y_k := Q^{-1} \cdot z_k\) est de Cauchy dans \(F\), complet par
hypothèse; si \(\ell\) désigne la limite de cette suite,
\(Q \cdot \ell\) fournit une limite à la suite des \(z_k\) dans \(G\).

Si l'on ajoute cette hypothèse -- inutile à ce stade, mais le plus
souvent très simple à vérifier -- au théorème, on peut en retirer une
autre, en général plus technique. En effet, un théorème dû à Stefan
Banach affirme que toute fonction linéaire continue entre deux espaces
de Banach qui est inversible à un inverse continue. Compte tenu de ce
résultat, on peut reformuler l'énoncé du théorème des fonctions
implicites en omettant la vérification de l'inversibilité de
\(\partial_y f\):

\hypertarget{TFI-2-alt}{%
\subsubsection{Théorème des Fonctions Implicites}\label{TFI-2-alt}}

Soient \(E\), \(F\) et \(G\) trois espaces vectoriels normés, \(F\) et
\(G\) étant complets, et \(f\) une fonction définie sur un ouvert \(W\)
de \(E \times F\) \[
f: (x, y) \in W \subset \mathbb{R}^n \times \mathbb{R}^m \to f(x, y) \in \mathbb{R}^m
\] qui soit continûment différentiable et dont la différentielle
partielle \(\partial_y f\) est inversible en tout point de \(W\). Si le
point \((x_0, y_0)\) de \(W\) vérifie \(f(x_0, y_0)= 0\), alors il
existe des voisinages ouverts \(U\) de \(x_0\) et \(V\) de \(y_0\) tels
que \(U \times V \subset W\) et une fonction implicite
\(\psi: U \to F\), continûment différentiable, telle que pour tous
\(x \in U\) et \(y \in V\), \[
f(x, y) = 0
\; \Leftrightarrow \; 
y = \psi(x).
\] De plus, la différentielle de \(\psi\) est donnée pour tout
\(x \in U\) par \[
d \psi(x) = - (\partial_y f(x, y))^{-1} \cdot \partial_x f(x, y) \, \mbox{ où } \, y=\psi(x).
\]

\hypertarget{todo-1}{%
\subsubsection{TODO}\label{todo-1}}

Nouvelle définition de la différentielle d'ordre supérieur,
compatibilité avec la dimension finie

\hypertarget{todo-2}{%
\subsubsection{TODO}\label{todo-2}}

Développement de Taylor, Taylor avec reste intégral, etc.

\hypertarget{todo-calcul-des-variations}{%
\section{TODO -- Calcul des
variations}\label{todo-calcul-des-variations}}

\hypertarget{todo-diffuxe9rentiation-dune-composition}{%
\subsubsection{TODO Différentiation d'une
composition}\label{todo-diffuxe9rentiation-dune-composition}}

Montrer conditions sous lesquelles on a (dans des espaces de fcts
continues): \[
d (f \mapsto g \circ f) 
=
h \mapsto (x \mapsto dg(f(x)) \cdot h(x))
\]

\vspace{3.25ex plus 1ex minus .2ex}\protect\hypertarget{todo-duxe9monstration}{}{\textbf{TODO
-- Démonstration}\quad}Insister sur le caractère nécessaire du résultat
(via composition du résultat avec \(f \to f(x)\), soit ``raisonner à
\(x\) fixé'').

Puis utiliser théorème des accroissements finis: on considère à \(x\)
fixé \[
h \in \mathbb{R}^n \mapsto g (f(x)+ h) - g(f(x)) - dg(f(x)) \cdot h
\] dont la différentielle vaut \(dg(f(x)+h) - dg(f(x))\), aussi petit
que l'on veut quand \(\|h\|\) est petit, et ce uniformément par rapport
à \(x\), ce qui donne en substituant \(h(x)\) à \(h\) (où \(h\) est
désormais une fonction bornée) \[
g (f(x)+ h(x)) = g(f(x)) + dg(f(x)) \cdot h(x) + o(\|h\|_{\infty}) 
\] où le \(o\) est uniforme par rapport à \(x\); et c'est
tout.\hfill$\blacksquare$

\hypertarget{lagrangien}{%
\subsubsection{Lagrangien}\label{lagrangien}}

Fonction \[
L: (x, y, y') \in U \subset \mathbb{R}\times \mathbb{R}^n \mapsto L(x, y, y') \in \mathbb{R}^n,
\] \(U\) ouvert, \(L\) supposée continûment différentiable.

\hypertarget{todo-3}{%
\subsubsection{TODO}\label{todo-3}}

(Nota: \(C^1\) est ``de confort'', on pourrait y arriver avec \(L\) cont
et diff partielles par rapport à \(y\) et \(y'\) continues) Plus tard,
nécessaire de renforcer régularité pour faire IPP et obtenir équation
d'Euler-Lagrange (suffit de supposer que \(\partial_{y'}L\) est diff /
\(x\) et que le résultat est cont).

\hypertarget{diffuxe9rentielle-dune-fonctionelle}{%
\subsubsection{Différentielle d'une
fonctionelle}\label{diffuxe9rentielle-dune-fonctionelle}}

\[
J(y) = \int_a^b L(x, y(x), y'(x) \, dx
\]

\[
dJ(y) \cdot h = \int_a^b \partial_{y}L(x, y(x), y'(x)) \cdot h(x)+
\partial_{y'} L(x, y(x), y'(x)) \cdot h'(x)
\, dx
\]

\hypertarget{todo-exercices}{%
\section{TODO -- Exercices}\label{todo-exercices}}

\hypertarget{todo-thuxe9oruxe8me-de-banach}{%
\subsection{TODO -- Théorème de
Banach}\label{todo-thuxe9oruxe8me-de-banach}}

\hypertarget{todo-compluxe9tude}{%
\subsection{TODO -- Complétude}\label{todo-compluxe9tude}}

Montrer que l'espace des fonctions continues de \([0,1]\) dans
\(\mathbb{R}\) n'est pas complet pour la norme \[
\|f\|_1 = \int_0^1 |f(x)| \, dx.
\]

\hypertarget{gradient-de-forme}{%
\subsection{Gradient de Forme}\label{gradient-de-forme}}

(dérivée une fonctionnelle dépendant de la géométrie en transportant la
géomtrie par \(I+u\), puis dérivée par rapport à \(u\) ?)

\(\Omega\) dans \(U\) paramétrisé par une déformation \(T = I + u\) avec
\(u\) petit et une base \(\Omega_0\) qui est un compact à bords \(C^1\).
Différencier le volume de \(\Omega\) par rapport \(T\) (chgt de
variable, etc), utiliser Stokes, montrer que la différentielle ne dépend
que de \(\left<V, n\right>\).

\hypertarget{thuxe9oruxe8me-de-cauchy-intuxe9gral}{%
\subsection{Théorème de Cauchy
Intégral}\label{thuxe9oruxe8me-de-cauchy-intuxe9gral}}

Montrer la version circulaire, sous l'hypothèse que \(f\) est de
différentielle \(\mathbb{C}\)-linéaire et continue (en faisant une chain
rule en dim infinie) \ldots{} mais est-ce vraiment nécessaire ? Ne
peut-on pas calculer les dérivées par rapport au centre et au rayon sans
ça, avec uniquement la dérivation ``point par point'' et les résultats
de convergence dans les intégrales ?

\end{document}
